% Dieser Text ist urheberrechtlich gesch�tzt
% Er stellt einen Auszug eines von mir erstellten Referates dar
% und darf nicht gewerblich genutzt werden
% die private bzw. Studiums bezogen Nutzung ist frei
% Nov. 2007
% Autor: Sascha Frank 
% Universit�t Freiburg 
% www.informatik.uni-freiburg.de/~frank/
\documentclass{beamer}
\setbeamertemplate{navigation symbols}{}


\graphicspath{ {pictures/} }


%\setbeamercolor{frametitle}{fg=black,bg=white}
%\setbeamercolor{title}{fg=black,bg=yellow!85!orange}
\usetheme{CambridgeUS}
% \usecolortheme[named=red]{structure}

\usepackage{ngerman}
\usepackage{ulem}
%\beamersetuncovermixins{\opaqueness<1>{25}}{\opaqueness<2->{15}}

% 1- Block title (background and text)
\setbeamercolor{block title}{bg=gray!70}
% 2- Block body (background)
\setbeamercolor{block body}{bg=gray!10}


\newcommand{\vq}{\mathbf{r}}
\newcommand*\diff{\mathop{}\!\mathrm{d}}  %text d in integrals
\newcommand{\iu}{{i\mkern1mu}} 	%imaginary unit
\newcommand{\me}{\mathrm{e}} 	%euler
\newcommand{\vj}{\mathbf{j}}
\newcommand{\vk}{\mathbf{k}}
\newcommand{\vp}{\mathbf{p}}
\newcommand{\vsigma}{\boldsymbol{\sigma}}
\newcommand{\FT}{\mathrm{FT}}

\newcommand{\real}{\mathrm{Re}}
\newcommand{\imag}{\mathrm{Im}}

\newcommand{\absatz}{\vskip3mm}


\begin{document}
\title{Numerik in der Festkörperphysik}
\subtitle{Propagation quantenmechanischer Wellenpakete}
\author[Phillipp Reck]{Dr. rer. nat. Phillipp Reck}
\institute[Probevorlesung: Wahlteil]{Probevorlesung TH Nürnberg, Numerik: Wahlteil}
\date{tba} 

\begin{frame}
\titlepage\end{frame}

\begin{frame}\frametitle{Die 5 Level des Lehrens (aus Sicht des Lernenden)}
    
\begin{enumerate}
  \item  \glqq Keine Ahnung wovon er spricht\grqq
  \item  \textcolor<2->{red}{\glqq Ich kann mir grob vorstellen, worum es geht\grqq}
  \item  \glqq Ich könnte es mit Hilfestellung anwenden\grqq
  \item  \glqq Ich könnte es ohne Hilfestellung anwenden\grqq
  \item  \glqq Ich könnte es jemand anderem beibringen\grqq
\end{enumerate}

\vfill

\visible<3->{
Vorraussetzung:
\begin{itemize}
  \item Differentialgleichungen/-systeme
  \item Fourier-Transformation
  \item Taylor-Entwicklung
\end{itemize}}
\end{frame} 
    

\begin{frame}
\frametitle{Inhaltsverzeichnis}\tableofcontents
\end{frame} 


\section{Motivation -- Ausgangssituation}
% \subsection{Physikalisch}


% \begin{frame}[t] %Time Reversal Mirrors
%  \frametitle{Time Reversal Mirrors}
%  \begin{alertblock}{Wahrscheinlich ändern:}
%   direkt das Video 
%   \end{alertblock}
% \begin{columns}
%   \column{0.55\textwidth}
%     \begin{figure}\centering
%      \includegraphics[width = \columnwidth]{TM_1.png}
%     \end{figure}
%   \column{0.45\textwidth}
%     \begin{figure}\centering
%      \includegraphics[width = \columnwidth]{TM_2.png}
%     \end{figure}
% \end{columns}
% \vspace{\fill}
% {\footnotesize \alert{ Derode A, Tourin A. and Fink M.}, J. Appl. Phys. \textbf{85}, 6343–52 (1999)}
% \end{frame}

\begin{frame}%Water principle
  \frametitle{Time mirror for water waves -- \\instantaneous time mirror (ITM)}
      \begin{figure}\centering
       \includegraphics[width = 0.7\columnwidth]{water1b.png}
      \end{figure}
  \vspace{\fill}
  {\footnotesize \alert{Bacot V., Labousse M., Eddi A., Fink M., Fort E.}, Nature Physics 12, 972 (2016)}
  \end{frame}
  
%   \begin{frame}%Water Example
%   \frametitle{Time mirror for water waves -- \\instantaneous time mirror (ITM)}
%       \begin{figure}\centering
%        \includegraphics[width = \columnwidth]{water2a.png}
%       \end{figure}
%   \vspace{\fill}
%   {\footnotesize \alert{Bacot V., Labousse M., Eddi A., Fink M., Fort E.}, Nature Physics 12, 972 (2016)}
%   \end{frame}

  
\begin{frame}[t]%Graphene Introduction
  \frametitle{Graphen}
 \begin{columns}
   \column{0.5\textwidth}
     \begin{figure}\centering
      \includegraphics[width = 0.8\columnwidth]{lattice.pdf}
     \end{figure} 
   \column{0.5\textwidth}
   \begin{figure}\centering
      \includegraphics[width = 0.8\columnwidth]{bands.png}
     \end{figure}
 \end{columns}
 \visible<2->{
 \begin{block}{Partielle Differentialgleichung -- Schrödingergleichung}
  %  \begin{center}
  \begin{align*}
  \iu \hbar \frac{\partial}{\partial t}  \psi(x, y, t) =& \hat{H} \psi(x, y, t) \\
  \uncover<3->{
    \iu \hbar \frac{\partial}{\partial t}  \psi(x, y, t) =& -\iu \hbar v_F \left(\frac{\partial}{\partial x}\sigma_x + \frac{\partial}{\partial y} \sigma_y \right) \psi(x, y, t)
  }
   \end{align*}
  % \end{center}
 \end{block}
 }
%  \vskip-0.5cm
%  \visible<4->{\begin{columns}
%    \column{0.55\textwidth}
%    $f(t) = \left\lbrace \begin{array}{ll} 
%        1, & \text{if } t_0 < t < t_0+\Delta t \\ 
%        0, & \text{otherwise} \end{array} \right.$
%    \column{0.45\textwidth}
%     \begin{figure} \centering 
%  %       analytical
%      \includegraphics[width=1\textwidth]{pulse_shape2}
%    \end{figure} 
%  \end{columns}} 
 \end{frame}

 \section{Numerische Problemstellung: AWP mit partieller DGL}
 
\begin{frame}[t] %Numerische Problemstellung
  \frametitle{Numerische Problemstellung}
 Quantenmechanische Elektron-Wellenfunktion: \\
 $\psi: \mathbb{R}^{n+1} \rightarrow \mathbb{C}^m \quad \text{$n$: Raum-Dimensionen, $m$: (Pseudo-)Spin-Dimension}$

 \absatz
 Hamilton-Operator (Differentialoperator): $\hat{H}(x, y, \frac{\partial}{\partial x}, \frac{\partial}{\partial y}, t)$

 \begin{block}{Anfangswertproblem (AWP) mit partiellem DGLS (Graphen: 1. Ordnung)}
  \begin{equation*}
    \iu \hbar \frac{\partial}{\partial t}  \psi(x, y, t) = \hat{H} \cdot \psi(x, y, t) 
  \end{equation*}

    \text{mit gegebenem }$\psi(x, y, t=0)$
    \end{block}
\end{frame}

\section{Schritte des numerischen Lösens}
\begin{frame}[t] %Diskretisieren auf Gitter
  \frametitle{Lösung von Teilproblemen Ia: Diskretisierung auf Raum-Gitter}

\begin{columns}
  \column{0.56\textwidth}
  \only<1>{\begin{figure}\centering
   \includegraphics[width = \columnwidth]{function_no_grid_auto.png}
  \end{figure}}
  \only<2->{\begin{figure}\centering
     \includegraphics[width = \columnwidth]{function_on_grid_auto.png}
    \end{figure} }
  \column{0.4\textwidth}
  \begin{align*}
    &&\psi(x) &= \quad e^{-x^2} \\
    \uncover<2->{
    \rightarrow &&\psi &= \left(
      \begin{array}{c}
        \vdots \\ 0.98202114\\ 0.9979862 \\ 0.9979862 \\  0.98202114 \\ 0.95085313 \\ 0.90594606 \\\vdots 
      \end{array}\right)
      }
  \end{align*}
    
\end{columns}

\end{frame}

\begin{frame}[t] %Diskretisieren auf Gitter
  \frametitle{Lösung von Teilproblemen Ia: Diskretisierung auf Raum-Gitter}
 Quantenmechanische Elektron-Wellenfunktion: \\
 \sout{$\psi: \mathbb{R}^{n+1} \rightarrow \mathbb{C}^m$} $ \quad \text{$n$: Raum-Dimensionen, $m$: (Pseudo-)Spin-Dimension}$
 \\
\textcolor{red}{$\psi: \mathbb{R} \rightarrow \mathbb{C}^{\text{Länge}^n\cdot m} $}

 \absatz
 Hamilton-Operator (Differentialoperator): $\hat{H}\textcolor{red}{(\frac{\partial}{\partial x}, \frac{\partial}{\partial y}, t)}$

 \begin{block}{Anfangswertproblem mit partiellem DGLS 1. Ordnung}
  \begin{equation*}
    \iu \hbar \frac{\partial}{\partial t}  \psi\textcolor{red}{(t)} = \hat{H} \cdot \psi\textcolor{red}{(t)} 
  \end{equation*}

    \text{mit gegebenem }$\psi\textcolor{red}{(t=0)}$
    \end{block}
\end{frame}

% \begin{frame}[t] %Diskretisieren auf Gitter
%   \frametitle{Lösung von Teilproblemen Ib: Diskretisierung auf Zeit-Gitter}
%  Quantenmechanische Elektron-Wellenfunktion: \begin{equation*}\psi: \mathbb{R}^{n+1} \rightarrow \mathbb{C}^m \quad \text{$n$: Raum-Dimensionen, $m$: (Pseudo-)Spin-Dimension}\end{equation*}

%  Hamilton-Operator (Differentialoperator): $\hat{H}(\frac{\partial}{\partial x}, \frac{\partial}{\partial y}, t)$

%  \begin{block}{Anfangswertproblem mit partiellem DGLS 1. Ordnung}
%   \begin{equation*}
%     \iu \hbar \frac{\partial}{\partial t}  \psi(x, y, t) = \hat{H} \cdot \psi(x, y, t) 
%   \end{equation*}

%     \text{mit gegebenem }$\psi(x, y, t=0) =\psi_0(x, y)$
%     \end{block}
% \end{frame}

\begin{frame}[t] %Räumliche Ableitung durch Fourier-Trafo
  \frametitle{Lösung von Teilproblemen II: Räumliche Ableitung durch Fourier-Transformationen}
  \vskip-1cm
  \begin{align*}
  \frac{\partial }{\partial x}f(x)
      &\uncover<2->{= \frac{\partial }{\partial x} \underbrace{\FT^{-1}\left[\: \FT[f](k) \, \right](x)}_{f(x)} }  \\
      &\uncover<3->{= \frac{1}{\sqrt{2\pi}} \frac{\partial }{\partial x}  \int e^{\pm \iu kx} \:\FT[f](k) \,dk }\\
      &\uncover<4->{= \frac{1}{\sqrt{2\pi}}  \int \pm \iu k\, e^{\pm \iu kx} \:\FT[f](k) \,dk }\\
      &\uncover<5->{=\pm \iu \:\FT^{-1} [k \:\FT[f](k)] (x) }\\
      % \text{\absatz}
      \uncover<6->{\frac{\partial^n }{\partial x^n}f(x)&=}\uncover<7->{(\pm \iu)^n \:\FT^{-1} [k^n \:\FT[f](k)] (x) }
  \end{align*}

  \visible<8->{ $\hat{H}(\frac{\partial}{\partial x}, \frac{\partial}{\partial y}, t) \rightarrow \hat{H}(k_x, k_y, t)$}
  
  \absatz
  \visible<9->{Algorithmus der Fast-Fourier-Trafo (FFT) wird verwendet.}

\end{frame}


\begin{frame}[t] %Zeitliche Ableitung durch formale Lösung
  \frametitle{Lösung von Teilproblemen IIIa: Zeitliche Ableitung durch formale Lösung}
 Quantenmechanische Elektron-Wellenfunktion: \begin{equation*}\psi: \mathbb{R}^{n+1} \rightarrow \mathbb{C}^m \quad \text{$n$: Raum-Dimensionen, $m$: (Pseudo-)Spin-Dimension}\end{equation*}

 Hamilton-Operator (Differentialoperator): $\hat{H}(k_x, k_y, t)$

 \begin{columns}
  \column{0.45\textwidth}
 \begin{block}{Formale Lösung}
  \begin{align*}
    \iu \hbar \frac{\partial}{\partial t}  \psi(t) = \hat{H} \cdot \psi(t)  \\
    \visible<5->{\Rightarrow  \psi(t) = e^{-\frac{\iu}{\hbar}\hat{H}t} \psi(0)}
  \end{align*}
    \end{block}
    
  \column{0.45\textwidth}
  \visible<2->{
  \begin{block}{Formale Lösung -- simple}
   \begin{align*}
     \frac{d}{d t}  &f(t) = a\cdot f(t)  \\
     \Rightarrow &f(t) = \visible<3->{ C\, e^{at}}\\
     \visible<4->{\Rightarrow &f(t) =  f(0)\, e^{at}}
   \end{align*}
     \end{block}}
     
  \end{columns}
  \visible<6->{Problem verschoben von Lösen eines DGLS zu Berechnen einer Matrix-Exponentialfunktion}
\end{frame}


\begin{frame}[t] %Berechnung der Matrix-Exponentialfunktion
  \frametitle{Lösung von Teilproblemen IIIb: Berechnung der Matrix-Exponentialfunktion}
  $e^{-\iu\hat{H}t}$ \pause ist definiert über Polynom-Darstellung:
  $e^{-\iu\hat{H}t }= \sum\limits_n a_n(-\iu t) P_n(\hat{H})$ 
  \pause
  \begin{columns}[t]
    \column{0.35\textwidth}
   \begin{block}{Taylor-Reihe}
    \begin{align*}
      e^{-\iu\hat{H}t} = \sum\limits_{n} \frac{(-\iu t)^n}{n!} \hat{H}^n
    \end{align*}
      \end{block}
      \pause
      Es stellt sich heraus, dass die Taylor-Reihe nicht besonders gut konvergiert.
    
      
    \column{0.55\textwidth}
    \visible<5->{
    \begin{block}{Tschebyschew-Polynome}
      Rekursion:
      \vskip-8mm
     \begin{align*}
      T_0(x) &= 1, \\
      T_1(x) &= x, \\ 
      T_n(x) &= 2xT_{n-1}(x)-T_{n-2}(x) 
    \end{align*}
    \visible<6->{
      Orthogonal mit Skalarprodukt:
      \vskip-8mm
      \begin{align*}
      \langle T_i, T_j \rangle = \int\limits_{-1}^{1} \frac{T_i(x)T_j(x)}{\sqrt{1-t^2}} \diff x = 0, i\neq j 
     \end{align*}}
     \vskip-8mm
       \end{block}}
       
    \end{columns}    
\end{frame}

\begin{frame}[t] %Lösung
  \frametitle{Gesamte Lösung}
  \vskip-5mm
  % \begin{block}{Anfangswertproblem (AWP) mit partiellem DGLS (Graphen: 1. Ordnung)}
    \begin{equation*}
      \iu \hbar \frac{\partial}{\partial t}  \psi(t) = \hat{H}\left(\frac{\partial}{\partial x}, \frac{\partial}{\partial y}, t\right) \cdot \psi(t) 
    \end{equation*}
  
      \text{mit gegebenem }$\psi(t=0)$.
      % \end{block}
    \begin{block}{Lösung (after some magic)}
  \begin{center}$\psi(t) = \sum\limits_{n=0}^N e^{-\frac{\iu}{\hbar}E_0 t} a_n(\frac{\Delta E t}{2\hbar}) T_n(-\iu\hat{H}_\text{norm})\psi(0)$\end{center}
    \end{block}     
  mit

    \begin{itemize}
  \item $T_n(x)$: Tschebyschew-Polynome
  \item $\hat{H}_\text{norm} = 2 \frac{\hat{H}-E_0}{\Delta E}$
  \item $E_0$, $\Delta E$: Skalierungsfaktoren 
  \item $a_n(x) = (2-\delta_{0,n})(-\iu)^n J_n(x)$
  \item $J_n(x)$: Bessel-Funktion der 1. Art
  \item $N= N(t) \approx 40 + 1.04 \frac{\Delta E t}{2\hbar}$
    \end{itemize}
\end{frame}

\section{Beispiel: Echo eines Wellenpakets in Graphen}
\begin{frame}[t]%
  \vskip-3mm
  \frametitle{Echo in Graphen}
     \begin{figure}\centering
      \includegraphics[width = 0.5\columnwidth]{hbar-Wellenpaket.png}
     \end{figure} 
 \end{frame}
















% \begin{frame}
% \frametitle{Titel} 
% Die einzelnen Frames sollte einen Titel haben
% \end{frame}
% \subsection{Unterabschnitt Nr.1.1  }
% \begin{frame} 
% Denn ohne Titel fehlt ihnen was
% \end{frame}


% \section{Abschnitt Nr. 2} 
% \subsection{Listen I}
% \begin{frame}\frametitle{Aufz\"ahlung}
% \begin{itemize}
% \item Einf\"uhrungskurs in \LaTeX  
% \item Kurs 2  
% \item Seminararbeiten und Pr\"asentationen mit \LaTeX 
% \item Die Beamerclass 
% \end{itemize} 
% \end{frame}

% \begin{frame}\frametitle{Aufz\"ahlung mit Pausen}
% \begin{itemize}
% \item  Einf\"uhrungskurs in \LaTeX \pause 
% \item  Kurs 2 \pause 
% \item  Seminararbeiten und Pr\"asentationen mit \LaTeX \pause 
% \item  Die Beamerclass
% \end{itemize} 
% \end{frame}

% \subsection{Listen II}
% \begin{frame}\frametitle{Numerierte Liste}
% \begin{enumerate}
% \item  Einf\"uhrungskurs in \LaTeX 
% \item  Kurs 2
% \item  Seminararbeiten und Pr\"asentationen mit \LaTeX 
% \item  Die Beamerclass
% \end{enumerate}
% \end{frame}
% \begin{frame}\frametitle{Numerierte Liste mit Pausen}
% \begin{enumerate}
% \item  Einf\"uhrungskurs in \LaTeX \pause 
% \item  Kurs 2 \pause 
% \item  Seminararbeiten und Pr\"asentationen mit \LaTeX \pause 
% \item  Die Beamerclass
% \end{enumerate}
% \end{frame}

% \section{Abschnitt Nr.3} 
% \subsection{Tabellen}
\begin{frame}\frametitle{Tabellen}
\begin{tabular}{|c|c|c|}
\hline
\textbf{Zeitpunkt} & \textbf{Kursleiter} & \textbf{Titel} \\
\hline
WS 04/05 & Sascha Frank &  Erste Schritte mit \LaTeX  \\
\hline
SS 05 & Sascha Frank & \LaTeX \ Kursreihe \\
\hline
\end{tabular}
\end{frame}


\begin{frame}\frametitle{Tabellen mit Pause}
\begin{tabular}{c c c}
A & B & C \\ 
\pause 
1 & 2 & 3 \\  
\pause 
A & B & C \\ 
\end{tabular} 
\end{frame}


% \section{Abschnitt Nr. 4}
% \subsection{Bl\"ocke}
\begin{frame}\frametitle{Bl\"ocke}

\begin{block}{Blocktitel}
Blocktext 
\end{block}

\begin{exampleblock}{Blocktitel}
Blocktext 
\end{exampleblock}


\begin{alertblock}{Blocktitel}
Blocktext 
\end{alertblock}
\end{frame}

% \section{Abschnitt Nr. 5}
% \subsection{Geteilter Bildschirm}

\begin{frame}\frametitle{Zerteilen des Bildschirmes}
\begin{columns}
\begin{column}{5cm}
\begin{itemize}
\item Beamer 
\item Beamer Class 
\item Beamer Class Latex 
\end{itemize}
\end{column}
\begin{column}{5cm}
\begin{tabular}{|c|c|}
\hline
\textbf{Kursleiter} & \textbf{Titel} \\
\hline
Sascha Frank &  \LaTeX \ Kurs 1 \\
\hline
Sascha Frank & \LaTeX \ Kursreihe \\
\hline
\end{tabular}
\end{column}
\end{columns}
\end{frame}

% \subsection{Bilder} 
\begin{frame}\frametitle{Bilder in Beamer}
\begin{figure}
% \includegraphics[scale=0.5]{PIC1} 
\caption{Die Abbildung zeigt ein Beispielbild}
\end{figure}
\end{frame}


% \subsection{Bilder und Listen kombiniert} 

\begin{frame}
\frametitle{Bilder und Itemization in Beamer}
\begin{columns}
\begin{column}{5cm}
\begin{itemize}
\item<1-> Stichwort 1
\item<3-> Stichwort 2
\item<5-> Stichwort 3
\end{itemize}
\vspace{3cm} 
\end{column}
\begin{column}{5cm}
\begin{overprint}
% \includegraphics<2>{PIC1}
% \includegraphics<4>{PIC2}
% \includegraphics<6>{PIC3}
\end{overprint}
\end{column}
\end{columns}
\end{frame}

% \subsection{Bilder die mehr Platz brauchen} 
\begin{frame}[plain]
\frametitle{plain, oder wie man mehr Platz hat}
\begin{figure}
% \includegraphics[scale=0.5]{PIC1} 
\caption{Die Abbildung zeigt ein Beispielbild}
\end{figure}
\end{frame}







\end{document}