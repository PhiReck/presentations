% Dieser Text ist urheberrechtlich gesch�tzt
% Er stellt einen Auszug eines von mir erstellten Referates dar
% und darf nicht gewerblich genutzt werden
% die private bzw. Studiums bezogen Nutzung ist frei
% Nov. 2007
% Autor: Sascha Frank 
% Universit�t Freiburg 
% www.informatik.uni-freiburg.de/~frank/
\documentclass{beamer}
\setbeamertemplate{navigation symbols}{}


\graphicspath{ {pictures/} }


%\setbeamercolor{frametitle}{fg=black,bg=white}
%\setbeamercolor{title}{fg=black,bg=yellow!85!orange}
\usetheme{CambridgeUS}
\usepackage{ngerman}
\usepackage{ulem}
%\beamersetuncovermixins{\opaqueness<1>{25}}{\opaqueness<2->{15}}

% 1- Block title (background and text)
\setbeamercolor{block title}{bg=gray!70}
% 2- Block body (background)
\setbeamercolor{block body}{bg=gray!10}


\newcommand{\vq}{\mathbf{r}}
\newcommand*\diff{\mathop{}\!\mathrm{d}}  %text d in integrals
\newcommand{\iu}{{i\mkern1mu}} 	%imaginary unit
\newcommand{\me}{\mathrm{e}} 	%euler
\newcommand{\vj}{\mathbf{j}}
\newcommand{\vk}{\mathbf{k}}
\newcommand{\vp}{\mathbf{p}}
\newcommand{\vsigma}{\boldsymbol{\sigma}}

\newcommand{\real}{\mathrm{Re}}
\newcommand{\imag}{\mathrm{Im}}


\begin{document}
\title{Numerik in der }
\subtitle{Am Beispiel X}
\author[Phillipp Reck]{Dr. rer. nat. Phillipp Reck}
\institute[Probevorlesung: Pflichtteil]{Probevorlesung TH Nürnberg, Numerik: Pflichtteil}
\date{tba} 

\begin{frame}
\titlepage
\end{frame}

\institute[Probevorlesung: Pflichtteil]{Probe\sout{\textcolor{red}{vorlesung}}vortrag TH Nürnberg, Numerik: Pflichtteil}


\begin{frame}\frametitle{Probevorlesung?}
    \pause
  \begin{enumerate}
    \item Zeit
    \item unterschiedliches Vorwissen im Publikum
  \end{enumerate}
  \end{frame} 

\begin{frame}
  \titlepage
  \end{frame}
  


\begin{frame}\frametitle{Die 5 Level des Lehrens (aus Sicht des Lernenden)}
    \pause
\begin{enumerate}
  \item  \glqq Keine Ahnung wovon er spricht\grqq\pause
  \item  \glqq Ich kann mir grob vorstellen, worum es geht\grqq\pause
  \item  \textcolor<7>{red}{\glqq Ich könnte es mit Hilfestellung anwenden\grqq}\pause
  \item  \glqq Ich könnte es ohne Hilfestellung anwenden\grqq\pause
  \item  \glqq Ich könnte es jemand anderem beibringen\grqq\pause
\end{enumerate}
\end{frame} 


\begin{frame}\frametitle{Nutzen der Numerik}
    \begin{block}{Numerik}
      Ein mathematisches Problem mit Hilfe des Computers näherungsweise zu lösen.
    \end{block}
    \pause
      \begin{itemize}
        \item (Lineare) Gleichungssysteme
        \item Integration
        \item Differentialgleichungen
        \item Interpolation, Fitting???
      \end{itemize}
    \pause

    BILD: Blackbox computer ---> eher nächste Seite
\end{frame} 


\begin{frame}\frametitle{Ist es sinnvoll, was der Computer berechnet?}
  \framesubtitle{Kondition, Stabilität, Konsistenz, Konvergenz}

   BILDER?
  \begin{block}{Kondition $\kappa$}
    Wie stark sich ein Fehler in der Problemstellung/den Eingangsdaten auf das Ergebnis auswirkt. \\
    (Unabhängig von Algorithmus)
  \end{block}
  \pause
  \begin{block}{Stabilität $???$}
    Wie stark wirkt sich ein Fehler in der Problemstellung/den Eingangsdaten auf das Ergebnis aus. \\
    (Abhängig von Algorithmus)
  \end{block}
  \pause
  
  \begin{block}{Konsistenz $???$}
    Wie gut ein Algorithmus bei perfekten Eingangsdaten ist. \\
    (Abhängig von Algorithmus)
  \end{block}
  \pause
  
  \begin{block}{Konvergenz $???$}
    Wie schnell liefert der Algorithmus ein brauchbares Ergebnis. \\
    (Abhängig von Algorithmus)
  \end{block}
\end{frame} 

% \section{Beispiel -- Problemstellung}

% \section{Thema -- Verallgemeinertes Problem}

% \section{Thema -- Kondition}

% \section{Thema -- Algorithmus}

% \section{Thema -- Stabilität, Konsistenz, Konvergenz}

% \section{Beispiel -- Lösung}















% \begin{frame}
% \frametitle{Titel} 
% Die einzelnen Frames sollte einen Titel haben
% \end{frame}
% \subsection{Unterabschnitt Nr.1.1  }
% \begin{frame} 
% Denn ohne Titel fehlt ihnen was
% \end{frame}


% \section{Abschnitt Nr. 2} 
% \subsection{Listen I}
% \begin{frame}\frametitle{Aufz\"ahlung}
% \begin{itemize}
% \item Einf\"uhrungskurs in \LaTeX  
% \item Kurs 2  
% \item Seminararbeiten und Pr\"asentationen mit \LaTeX 
% \item Die Beamerclass 
% \end{itemize} 
% \end{frame}

% \begin{frame}\frametitle{Aufz\"ahlung mit Pausen}
% \begin{itemize}
% \item  Einf\"uhrungskurs in \LaTeX \pause 
% \item  Kurs 2 \pause 
% \item  Seminararbeiten und Pr\"asentationen mit \LaTeX \pause 
% \item  Die Beamerclass
% \end{itemize} 
% \end{frame}

% \subsection{Listen II}
% \begin{frame}\frametitle{Numerierte Liste}
% \begin{enumerate}
% \item  Einf\"uhrungskurs in \LaTeX 
% \item  Kurs 2
% \item  Seminararbeiten und Pr\"asentationen mit \LaTeX 
% \item  Die Beamerclass
% \end{enumerate}
% \end{frame}
% \begin{frame}\frametitle{Numerierte Liste mit Pausen}
% \begin{enumerate}
% \item  Einf\"uhrungskurs in \LaTeX \pause 
% \item  Kurs 2 \pause 
% \item  Seminararbeiten und Pr\"asentationen mit \LaTeX \pause 
% \item  Die Beamerclass
% \end{enumerate}
% \end{frame}

\section{Abschnitt Nr.3} 
\subsection{Tabellen}
\begin{frame}\frametitle{Tabellen}
\begin{tabular}{|c|c|c|}
\hline
\textbf{Zeitpunkt} & \textbf{Kursleiter} & \textbf{Titel} \\
\hline
WS 04/05 & Sascha Frank &  Erste Schritte mit \LaTeX  \\
\hline
SS 05 & Sascha Frank & \LaTeX \ Kursreihe \\
\hline
\end{tabular}
\end{frame}


\begin{frame}\frametitle{Tabellen mit Pause}
\begin{tabular}{c c c}
A & B & C \\ 
\pause 
1 & 2 & 3 \\  
\pause 
A & B & C \\ 
\end{tabular} 
\end{frame}


\section{Abschnitt Nr. 4}
\subsection{Bl\"ocke}
\begin{frame}\frametitle{Bl\"ocke}

\begin{block}{Blocktitel}
Blocktext 
\end{block}

\begin{exampleblock}{Blocktitel}
Blocktext 
\end{exampleblock}


\begin{alertblock}{Blocktitel}
Blocktext 
\end{alertblock}
\end{frame}

\section{Abschnitt Nr. 5}
\subsection{Geteilter Bildschirm}

\begin{frame}\frametitle{Zerteilen des Bildschirmes}
\begin{columns}
\begin{column}{5cm}
\begin{itemize}
\item Beamer 
\item Beamer Class 
\item Beamer Class Latex 
\end{itemize}
\end{column}
\begin{column}{5cm}
\begin{tabular}{|c|c|}
\hline
\textbf{Kursleiter} & \textbf{Titel} \\
\hline
Sascha Frank &  \LaTeX \ Kurs 1 \\
\hline
Sascha Frank & \LaTeX \ Kursreihe \\
\hline
\end{tabular}
\end{column}
\end{columns}
\end{frame}

\subsection{Bilder} 
\begin{frame}\frametitle{Bilder in Beamer}
\begin{figure}
% \includegraphics[scale=0.5]{PIC1} 
\caption{Die Abbildung zeigt ein Beispielbild}
\end{figure}
\end{frame}


\subsection{Bilder und Listen kombiniert} 

\begin{frame}
\frametitle{Bilder und Itemization in Beamer}
\begin{columns}
\begin{column}{5cm}
\begin{itemize}
\item<1-> Stichwort 1
\item<3-> Stichwort 2
\item<5-> Stichwort 3
\end{itemize}
\vspace{3cm} 
\end{column}
\begin{column}{5cm}
\begin{overprint}
% \includegraphics<2>{PIC1}
% \includegraphics<4>{PIC2}
% \includegraphics<6>{PIC3}
\end{overprint}
\end{column}
\end{columns}
\end{frame}

\subsection{Bilder die mehr Platz brauchen} 
\begin{frame}[plain]
\frametitle{plain, oder wie man mehr Platz hat}
\begin{figure}
% \includegraphics[scale=0.5]{PIC1} 
\caption{Die Abbildung zeigt ein Beispielbild}
\end{figure}
\end{frame}







\end{document}